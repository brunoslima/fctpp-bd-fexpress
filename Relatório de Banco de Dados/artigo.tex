\documentclass[10pt, twocolumn]{article}

\usepackage[brazil]{babel}
\usepackage[utf8]{inputenc}
\usepackage{graphicx}



\title{NoSQL\\Armazenamento e Recuperação de Dados Não-Relacionais}

\author{
Universidade Estadual Paulista\\
Faculdade de Ciências e Tecnologia \\ Departamento de Matemática e Computação\\
Bruno Santos de Lima\\
Leandro Ungari Cayres - leandroungari@gmail.com\\
\\
Presidente Prudente, SP - Brazil \\
}

\begin{document}

\maketitle
%--------------------------------------------------------
\begin{abstract}
Este trabalho apresenta o desenvolvimento, em plataforma Web, de um sistema unificado de gerenciamento de informações habitacionais referentes à cidade de Porto Alegre. Tem como objetivo promover uma organização dos dados de forma mais qualificada, assim é apresentada a atual situação bem como as principais caracteristicas deste sistema.
\end{abstract}


%--------------------------------------------------------
\section{Introdução}
\label{sect:introduction}
O gerenciamento de informações habitacionais consiste em armazenar e gerenciar dados sobre habitação de uma área, o qual está sob responsabilidade do Departamento de Habitação (DEMHAB), sendo feito de forma descentralizada, dificultando consultas e geração de relatórios.
 
Mediante a isso, foi proposta a criação do SGIHAB, inicialmente como um projeto piloto interno ao DEMHAB e a Coordenação de Urbanização (CUR), para posteriormene ser difundido aos outros departamentos.

O objetivo principal deste trabalho é desenvolver uma solução capaz de armazenar e organizar um conjunto de informações, em plataforma web, além de migrar de pequenas soluções proprietárias para ferramentas de software livre.

O  relatório está divido da seguinte forma: na Seção \ref{sect:gia}, é apresentado um rápido histórico sobre o departamento e a coordenação, assim como sobre os sistemas para o gerenciamento de informações e um levantamento de requisitos. Na Seção \ref{sect:msga}, é descrito o processo de migração do sistema, são apresentadas as tecnologias utilizadas, a descrição dos cadastros essenciais, a técnica de modelagem, a metodologia, demonstração do sistema em funcionamento e a avaliação dos usuários finais através de questionários, assim como os respectivos resultados.

Por fim, na Seção \ref{sect:conclusion}, são avialiados os resultados obtidos e as ferramentas utilizadas, também são apresentadas as limitações do sistema e um projeção dos trabalhos futuros.

%--------------------------------------------------------
\section{Gerenciamento de Informações de Áreas}
\label{sect:gia}

\subsection{O Departamento Municipal de Habitação}
\label{sect:dmh}
O Departamento de Habitação é uma autarquia	responsável pela gestão da política habitacional de interesse social, competindo-lhe o desenvolvimento do plano municipal, em que propõe diretrizes, objetivos e metas estratégicas de ação na superação do déficit habitacional da cidade.

Este departamento foi criado em 1965, a partir da reformulação do Departamento Municipal da Casa Popular, cujo propósito era extinguir atividades improvisadas e dedicar-se à execução de projetos organizados. Desde então, surgiram diversas parcerias de apoio financeiro e também políticas habitacionais; atualmente, através do Governo Federal, o Ministério das Cidades define os rumos da política habitacional e da gestão das cidades.

Pertence ao departamento, a Coordenação de Urbanização, consiste no setor responsável pela regularização urbanística das áreas integrantes da Política Habitacional do Município de Porto Alegre.

\subsection{Sistemas Legados para Gerenciar Informações de Áreas}
\label{sect:slgia}
O conjunto de dados sobre intervenções do DEMHAB sobre áreas é desentralizada, em que cada setor possui informações em meio impresso ou digital, em diversos formatos.
Na coordenação, tal situação não difere muito, muitas informações ainda estão em papel pois são muito antigas, ou outros projetos, não sendo integralmente digitais.

Os arquivos com informações adicionais são do tipo de texto, planilha e apresentação; relacionam-se aos arquivos CAD (\textit{Computer Aided Design}), como cálculos de áreas, histórico, fotos e entre outros.

Em relação ao banco de dados, este constitui o principal sistema legado para este projeto. A princípio, o objetivo desta base de dados era organizar informações relacionadas ao Programa de Regularização Fundiária. Posteriormente, foram incluídas novas informações sobre áreas em intervenção, consequentemente, gerando um grande números de relatórios para os processos administrativos.

No entanto, devido a crescente demanda de novos projetos aglutinada a renovação do conjunto de máquinas da CUR, alguns problemas surgiram, tais como: alto custo de manutenção devido ao uso de software proprietário e diversos problemas de compatibilidade com versões mais antigas, a partir disso iniciou-se um processo de restrições contra o uso de softwares proprietários e a proliferação de pequenos sistemas  

\subsection{Requisitos para o Gerenciamento de Áreas}
\label{sect:rga}
Este sistema possui um conjunto de requisitos que precisam ser geridos para que, além de disponibilizar informações já apresentadas pelos sistemas legados, possam ser adicionadas melhorias significativas.

Para o controle de acesso ao sistema é necessária a criação de um módulo que vincule informações sobre áreas diretamente aos seus usuários. A gestão destas áreas sempre foi feita através da estrutura de diretórios do sistema operacional em conjunto com um banco de dados, com limitações em termos de anexação de arquivos de mídia e texto. Desta forma, fez-se necessário um aprimoramento deste sistema, realizando a migração das informações de áreas já disponíveis.

Em relação ao controle de programas e projetos, estes eram organizados em documentos de textos e planilhas, o que compunha um obstáculo para a atualização destas informações, assim, foi realizado vínculo destas com informações sobre áreas.

O uso da plataforma traz inúmeras facilidades pois seu acesso necessita somente de um navegador, o que gera portabilidade entre os diferentes sistemas operacionais. Muitas características dos antigos sistemas foram trazidos para uma migração de plataforma com maior facilidade, porém, a nova plataforma permite uma padronização do tamanho da tela do sistema independente da resolução do monitor, já que páginas geradas em HTML (\textit{HyperText Markup Language}) são mais dinâmicas e se adaptam a diferentes resoluções.

%--------------------------------------------------------
\section{Migração do Sistema de Gerenciamento de Áreas}
\label{sect:msga}

\subsection{Tecnologias usadas no Projeto}
\label{sect:tup}
Como arquitetura de software foi utilizado o padrão MVC (\textit{Model - View - Controller}). O primeiro componente consiste no objeto da aplicação, o segundo como a aplicação será apresentada na tela e o terceiro define como a interface do usuário reagirá as suas entradas.

Esta abordagem é muita utilizada em sistemas web, na qual são separadas as \textit{Views} e os \textit{Models}, interligadas pelos \textit{Controllers}, desta forma, é possível que para diferentes interações do usuário com o sistema resultem em variadas combinações de \textit{Views}.

Para a implementação desta arquitetura, faz-se necessário o uso de uma linguagem de programação. O uso da Orientação a Objetos traz uma maior proximidade com objetos e ações do mundo real, desta forma, foi escolhido o PHP(\textit{PHP Hypertext Preprocessor}) por ser compatível com os principais sistemas operacionais, além de possuir semelhança de sintaxe com as linguagens C e C++, e também prover suporte a vários bancos de dados.

Perante a essa variedade de banco de dados compatíveis, é importante escolher que seja capaz de armazenar informações de forma segura e duradoura, assim optou-se pela escolha de um sistema de gerenciamento de banco de dados (SGBD) gratuito e de código aberto, o MySQL, por se tratar da base de dados mais utilizada no mundo para soluções web, sendo utilizado por grandes corporações mundiais.

Para a modelagem e diagramação deste sistema foi utilizada a ferramenta Argo UML, a qual suporta variados tipos de diagramas tais como casos de uso, classes e estados; e que também possibilita a exportação para linguagens como PHP, C/C++, Java e SQL.

Por fim, a utilizadação de um ambiente integrado de desenvolvimento (IDE) é fundamental para o desenvolvimento de um software de forma ágil, auxiliando na escrita, compilação e debugação de aplicações, além de contar com bibliotecas e plugins que otimizam o desenvolvimento, assim, foi escolhido o ambiente integrado NetBeans.

\subsection{Cadastros}
\label{sect:cad}
O sistema é composto de vários cadastros, que utilizam a ideia de listas que se assemelham a planilhas; e sua importância pode ser medida no nível de utilização.

Como primeiro grande conjunto de cadastros temos o de usuários, através deste é possível verificar quais pessoas estão responsabilizadas por uma determinada região, também possui informações pessoais ao funcionários, na qual somente os próprios tem acesso e também nesta seção é definido o nível hierárquico do sistema, definindo os administradores e as áreas de acesso limitadas a cada cargo dentro do gerenciador.

O próximo conjunto é o de cadastro de áreas, o qual permite verificar todas as informações sobre determinada área, tais como informações históricas, intervenções realizadas e futuras, assim o tipo de área e como esta é ocupada.

Também é possível destacar o cadastro de programas e projetos, o qual fornece informações em forma de relatórios de áreas subdivididos por programa e/ou projetos, apresentados na forma de lista e constitui uma novidade em comparação aos sistemas legados.

\subsection{Relatórios}
\label{rect:rel}
Os relatórios constituem uma importante ferramenta do gerenciador, através destes temos informações melhor organizadas para a realização de consultas, informar processos, atender requisições externas ou de instâncias superiores. Estes relatórios estão disponíveis nos seguintes tipos: relatórios de usuários, áreas, regiões de orçamento participativo, topografias, EVU's e programas e projetos.

\subsection{Modelagem}
\label{rect:model}
Para o desenvolvimento deste sistema foram estabelecidos dois tipos de diagramas, classes e casos de uso. O primeiro, são apresentadas todas as classes que foram implementadas, estas determinam as regras de acesso às classes de negócio e duas respectivas regras, enquanto no segundo são apresentados as regras de uso dos métodos para usuário e administrador, e somente administrador. 

\subsection{Desenvolvimento}
\label{rect:desen}
O desenvolvimento deste projeto foi baseado na metodologia ágil, esta surgiu na década de 90 como uma forma de reação aos métodos tradicionais, caracterizados pelo excessivo na criação de uma documentação completa. O projeto foi dividido em quatro iterações com um tempo médio de um mês cada.

A essência da metodologia ágil de desenvolvimento consiste em utilizar-se de regras leves porém que cumpram com o que foi exigido, permitindo ao mesmo tempo ser manipulável e consistente. Deve ser composta por uma pequena equipe, uso de especialistas, pequenos incrementos entre os testes e o uso de desenvolvedores experientes. Consiste em uma abordagem recente, 2002, baseada no desenvolvimento incremental, cooperativo e adaptativo.

A primeira iteração compreendeu um conjunto reduzido de informações mais relevantes sobre as áreas de intervenção. As classes que controlavam o acesso dos usuários ainda não foram implementas e mesmo com o sistema em funcionamento, isto não apresentou problemas ao gerenciamento e aos dados.

Na segunda iteração foram atribuidas mais características a classe área, foi modificado o relacionamento entre esta e a classe bairro, por fim foi implementada a classe topografia.

Na terceira iteração, foi adicionadas classes referentes ao cadastro de usuários no sistema, que em suas primeiras versões, ficará com acesso limitado aos administradores e em versões futuras, fora do escopo deste trabalho, permitirá que os próprios usuários alterem seus dados.

Por último, a quarta iteração contemplou classes referentes ao cadastro de programas e projetos e tipologias habitacionais. A princípio era previsto a finalização do cadastro de usuários, com a liberação dos respectivos perfis e a implementação de upload de arquivos, porém estas se demonstraram mais complexas do que o previsto, o que caracterizou o processo adaptativo desta metodologia de desenvolvimento.

\subsection{Uso do Sistema}
\label{rect:us}
Nesta seção são apresentadas as telas do sistema em funcionamento, assim como os procedimentos de visualização, cadastro e edição em algumas listas de informações e registros.

Em primeiro lugar, a visualização das informações em todas as seções é feita através do uso de listas e em alguns casos também é disponível a visualização completa, para uma melhor compreensão. Esta seção está dividida em: visualizando informações sobre áreas, regiões de orçamento participativo, EVU's, topografia, programas e projetos, e usuários.

A posteriori, a seção de cadastro permite a inserção de novas informações sobre os setores listados no parágrafo anterior, os quais por sua vez podem ser atualizados e/ou corrigidos através da seção de edição de informações.

\subsection{Avaliação dos Usuários}
\label{rect:aval}
Durante o processo de desenvolvimento foram realizadas avaliações restritas a dois setores do departamento, sendo realizadas de duas formas: através do retorno direto ao desenvolvedor no uso diário e através de questionários feitos principalmente para avaliar a nova interface que o SGIHAB proporcionou, sendo feitas através de questões alternativas, mas também disponibilizando espaço para comentários permitindo sugestões para a melhoria do sistema. Estes foram realizados através da ferramenta de formulários \textit{Google Docs} e enviados por e-mail ao usuário.

As questões revelaram uma aceitação acima de 90\% do sistema e através dos comentários foram relatadas melhorias em relação aos menus do sistema, novas informações e integração com outros sistemas.

%--------------------------------------------------------
\section{Conclusão}
\label{sect:conclusion}
Desde o princípio, a grande motivação para a realização desse projeto é o fato de haver uma necessidade real do sistema com usuários reais.

Como principais resultados podemos destacar o alcance que o uso da plataforma web proporcionou aos usuários por ser compatível com os diferentes sistemas operacionais, a simplicidade da interface de telas padronizadas e a possibilidade de vinculação de informações com os diferentes tipos de cadastros.

Este sistema traz inúmeras melhorias em relação aos sistemas legados, porém ainda apresentados limitações, dentre as quais podemos citar: a não integração com outros sistemas do departamento, a não agregação de conteúdos de mídia e arquivos de textos e planilhas aos diversos cadastros e a não visualização de dados utilizando serviços de mapas já conhecidos como \textit{Google Maps} e \textit{OpenStreetMap}.

As ferramentas utilizadas no desenvolvimento deste trabalho são amplamente usadas por analistas e desenvolvedores para facilitar a criação de sistemas, sendo algumas destas conhecidas de outros trabalhos realizados.

Como projeção para trabalhos futuros no SGIHAB, temos o intuito de incluir diversas sugestões dos usuários e implementar novos atributos as classes referentes às áreas e as regiões de orçamento participativo. 

Dois novos cadastros fundamentais também serão realizados para o aperfeiçoamento do sistema: um cadastro para gerenciar imagens e outro para o gerenciamento de mapas. O primeiro permitirá a inclusão das inúmeras imagens que antes eram armazenadas sob a estrutura de diretórios do sistema operacional, enquanto o segundo permitirá a inclusão de dados georreferenciados para a delimitação das áreas. Outra ferramenta que será aprimorada é a geração de relatórios, que atualmente gera relatórios fixos, sendo aprimorado para permitir a elaboração de relatórios mais personalizados.

Para a realização de todas essas novidades listadas acima, será de suma importância o uso de um \textit{framework} de desenvolvimento voltado para a linguagem PHP e o padrão MVC, permitindo a inclusão de funcionalidades já prontas em um menor tempo de codificação e, consequentemente, o tempo das iterações de desenvolvimento.

\end{document}