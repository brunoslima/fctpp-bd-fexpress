\documentclass[12pt, onecolumn, titlepage]{article}

\usepackage[brazil]{babel}
%\usepackage[latin1]{inputenc}
\usepackage[utf8]{inputenc}
\usepackage{graphicx}

\begin{document} %Inicio do documento

\begin{titlepage} %Capa
	
	\vfill
	\begin{center}
	
		{\large \textbf{Faculdade de Ciências e Tecnologia\\Universidade Estadual Paulista\\``Júlio de Mesquita Filho''}} \\[3cm]
		{\small \textbf{Bruno Santos de Lima - RA: 141251093}}\\
		{\small \textbf{Leandro Ungari Cayres - RA: 141250992}}\\[3cm]
		{\Large Definição do Projeto}\\
		{\Large Sistema Fexpress}\\[3cm]

	\hspace{.45\textwidth} %posiciona a minipage
	\begin{minipage}{.5\textwidth}
		\small Disciplina de Banco de Dados I. Professor Dr. Ronaldo Celso Messias Correia pelo curso de Ciência da Computação. Com data de entrega 05 de Junho de 2016 \\[0.5cm]
	\end{minipage}

	\vfill
	\vspace{1.5cm}
	
	\large \textbf{Presidente Prudente\\}
	\large \textbf{Maio - 2016}
	
	\end{center}
	
\end{titlepage}
%Fim da capa
\newpage

\renewcommand{\contentsname}{Índice}
\tableofcontents

\newpage

\section{Introdução}
\label{sect:introducao}

Este documento tem como proposito descrever detalhes referentes ao Sistema Fexpress, detalhes tanto no sentido de funcionamento do sistema, mas principalmente detalhes de caráter técnico relacionados a área de banco de dados. 

Esta coletânea esta dividida como segue: na seção \ref{sect:especificacao} é especificado o sistema bem com sua lógica de funcionamento, na seção \ref{sect:conceitual} é apresentado o modelo entidade relacionamento do sistema, a seção \ref{sect:relacional} mostra o modelo conceitual, em seguida a seção \ref{sect:normalizacao} evidencia discussões referentes ao processo de normalização, por fim a seção \ref{sect:algebra} apresenta detalhes pertinentes a Álgebra Relacional.

\section{Especificação do Problema}
\label{sect:especificacao}

O Sistema Frexpress, acrônimo de Food Express, consistem em um Sistema para gerenciamento de uma distribuidora de alimentos, com o intuito de armazenar os dados necessários para a que a empresa possa realizar a gestão de seus clientes, fornecedores, produtos e rotas de viagens, bem como fornecer um mecanismo que facilite a obtenção dos melhores caminhos para o transporte das mercadorias até as localizações de seus clientes.

No sistema os clientes (mercados e estabelecimentos comerciais do setor alimentício) da distribuidora podem realizar encomendas para serem entregues em seus respectivos estabelecimento, os gerentes da distribuidora devem ter acesso a essas encomendas e utilizam o sistema para elaborar viagens com rotas convenentes em termos de custos de viagem, especificando tipo de carga, veículo e motorista que faram parte da viagem. O Gerentes da distribuidora também tem controle de todo o sistema de estoque, podendo realizar pedidos aos seus fornecedores (fabricas, produtores rurais e empresas) para repor o mesmo, ou para atender as necessidades de um cliente em especifico.

Trata-se de um sistema web, implementado utilizando a linguagem de programação PHP, além da utilização da linguagem de marcação HTML, CSS, contendo algumas aplicações de JavaScript, além de utilizar como Sistema Gerenciador de Banco de Dados (SGBD) o MySQL.

\subsection{Entidades}
\label{sect:entidades}

Abaixo segue especificado o conjunto de entidades necessarias para a elaboração deste projeto, todas as entidades estão organizadas com o prefixo E seguido de um número que identifica a mesma de forma única, além disso cada entidade contém um conjunto de atributos que também estão especificados.

\begin{description}

\item E1. Funcionario
\item \qquad Atributos: 

\item E1.1 Motorista
\item \qquad

\item E1.2 Gerente
\item \qquad

\item E2. Fornecedor
\item \qquad

\item E3. 

\end{description}

\subsection{Relacionamento entre Entidades}
\label{sect:relacionamento}

\section{Esquema Conceitual}
\label{sect:conceitual}


\section{Esquema Relacional}
\label{sect:relacional}


\section{Normalização}
\label{sect:normalizacao}


\section{Álgebra Relacional}
\label{sect:algebra}


\end{document}